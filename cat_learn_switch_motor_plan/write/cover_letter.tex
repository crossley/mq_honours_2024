\documentclass[10pt]{article}
\usepackage{graphicx}
\usepackage{parskip}
\usepackage{geometry}
\geometry{
  top=20mm,
}
\usepackage{fancyhdr}
\lhead{\includegraphics[height=2.0cm]{../figures/MQ_MAS_HOR_RGB_POS.png}\vspace{0.0cm}}
\rhead{\includegraphics[width=6.0cm]{../figures/UC_Santa_Barbara_Wordmark_Navy_RGB.png}\vspace{0.5cm}}
\setlength\headheight{2cm}

\usepackage[hang, symbol]{footmisc}
\setlength\footnotemargin{10pt}
\renewcommand{\thefootnote}{\fnsymbol{footnote}}

\usepackage{hyperref}

\pagenumbering{gobble}

\begin{document}
\thispagestyle{fancy}
Dear Editors,

Please consider our enclosed manuscript titled
\textit{``Unique motor plans facilitate learning during task
switching, but at the expense of greater switch costs''} for
publication in \textit{Memory \& Cognition}. Our manuscript
is important because it is the among the first to
simultaneously investigate two fundamentally important
aspects of cognition and action: task switching and initial
task learning. 

We show that in attention-demanding environments, learning
is better when each task has unique motor responses compared
to when the same motor responses are used for both tasks,
but this improved learning comes at the price of increased
switch cost. We further show that in order to account for
this result, standard models of cognitive control must be
augmented with a novel motor control mechanism in which
there is competition among competing response options.
Finally, we show that switching between tasks that load on
the same memory system is easier than switching between
tasks that load on different memory systems, and that
switching between rule-based tasks that preferentially rely
on declarative memory is the easiest of all.

For expert evaluation of the task switching and cognitive
control and modelling elements of our manuscript, we
recommend the following reviewers:

\begin{itemize}
    \item Anne Collins; \href{mailto:annecollins@berkeley.edu}{annecollins@berkeley.edu}
    \item Scott Brown; \href{mailto:scott.brown@newcastle.edu.au}{scott.brown@newcastle.edu.au}
    \item Andrew Heathcote; \href{mailto:andrew.heathcote@newcastle.edu.au}{andrew.heathcote@newcastle.edu.au}
\end{itemize}

For expert evaluation of the initial task learning elements
of our manuscript, we recommend the following reviewers:

\begin{itemize}
    \item Carol A. Seger; \href{mailto:Carol.Seger@colostate.edu}{Carol.Seger@colostate.edu}
    \item S\'ebastien H\'elie; \href{mailto:shelie@purdue.edu}{shelie@purdue.edu}
\end{itemize}

Thank you for your consideration and we look forward to
hearing from you in due course.

Sincerely,

Matthew J. Crossley\footnote{
  Corresponding author:\\
  Matthew J. Crossley\\
  matthew.crossley@mq.edu.au\\
  School of Psychological Sciences\\
  Centre for Elite Performance, Expertise \& Training (CEPET)\\
  Australian Hearing Hub\\
  16 University Ave\\
  Macquarie University, NSW 2109, Australia
}\\
J. Vincent Filoteo\\
W. Todd Maddox\\
F. Gregory Ashby
\end{document}
